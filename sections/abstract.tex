%% LaTeX2e class for seminar theses
%% sections/abstract_en.tex
%% 
%% Karlsruhe Institute of Technology
%% Institute for Program Structures and Data Organization
%% Chair for Software Design and Quality (SDQ)
%%
%% Dr.-Ing. Erik Burger
%% burger@kit.edu
%%
%% Version 1.0.2, 2020-05-07
\textbf{Abstract[Deutsch].} Wissensgewinnung ist eine essentielle Eigenschaft des meschlichen Wesens, die uns von den anderen Kreaturen unterscheidet. Ein Instrument für Einordnung und Übertragung des Wissens, das nicht nur von Menschen, sondern auch von Computern verstanden werden kann, sind Ontologien. In dieser Proseminar-Arbeit werden die Ontologien in Software-Engineering-Meta-Forschung untersucht und analysiert, um die fundamentalen Kenntnisse in diesem wissenschaftlichen Bereich zu erwerben. Darüber hinaus haben Ontologien den Bezug auf semantische Suche, die in fast allen modernen Suchmaschinen wie Google, Bing, Yandex etc. verwendet wird. Das Ziel dieser Proseminar-Arbeit ist jedoch die Untersuchung von wissenschaftlichen Suchmaschinen mit dem Fokus auf die Anwendung der Ontologien in ihnen und wie sie zur Software-Engineering-Meta-Forschung beitragen könnten.
\newline\newline

\textbf{Abstract.} Retrieving Knowledge is an essential part of human being that distinguishes us from other creatures. One of the instrument for organizing and transferring knowledge that is understandable not only to humans, but to computers as well, is an ontology. This seminar work aims at investigating ontologies in Software-Engineering-Meta-Research and trying to analyze the retrieved information, in order to grasp an understanding and awareness of fundamentals in this scientific field. Moreover, ontologies are closely related to the Semantic search that is used by every contemporary search engine such as Google, Bing, Yandex etc. The purpose of this seminar was, however, about the investigation of scientific search engines with focus on applying ontologies in them and how they could contribute to the Software-Engineering-Meta-Research.       