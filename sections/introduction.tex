%% LaTeX2e class for seminar theses
%% sections/content.tex
%% 
%% Karlsruhe Institute of Technology
%% Institute for Program Structures and Data Organization
%% Chair for Software Design and Quality (SDQ)
%%
%% Dr.-Ing. Erik Burger
%% burger@kit.edu
%%
%% Version 1.0.2, 2020-05-07

\section{Introduction}
\label{ch:Introduction}
The term ontology originates from philosophy. Its etymology gives us the direct definition i.e. a science about being. More broadly, ontology is \flqq{a branch of metaphysics concerned with the nature and relations of being}\frqq\cite{OntPh}. But how does it relate to Computer Science? As far as we know, Information and Computer science are very precise and cannot admit a vague descriptions, as it could be met in philosophy. Thus, ontology in Computer Science is \flqq{a list of concepts and categories in a subject area that shows the relationships between them}\frqq\cite{OntCS}. This means simply the supplement of a common language, in order to facilitate general understanding of knowledge in machine way. 

The term Meta-research is \flqq{the use of scientific methodology to study science itself}\frqq\cite{MR}. This means that researchers are trying to understand the research itself i.e. how other researches should be conducted, what practices in the research are the most effective and in what research fields they should be used. 

These two terms and especially their combination in Software Engineering, namely how ontologies can contribute to the Meta-research and how Meta-research changes the ontology-based approaches in Software Engineering are part of the rather young scientific field that arouses a strong interest in it. Moreover, it is not a secret that ontology-based systems are actively used in the contemporary search mechanism. Therefore, it attracts an attention, how exactly ontologies are used not only in the traditional search, but also in the scientific one.

All of the topics named above have been investigated with two types of papers searches. For Ontologies in Software-Engineering-Meta-Research was used the reference-based search or so-called \flqq{Snowballing}\frqq\cite{Woh14} with seeds given by my advisor. Since the second main part, namely Ontologies in the scientific search, was not originally a topic of this seminar, there were used both the traditional database search with the keywords from the name of the topic as well as the ones given by my advisor, and the already mentioned reference-based search with seeds received from the first search method.

The remainder of this seminar work is structured as follows. Section 2 sets the necessary foundations to the terms ontologies and Meta-research; Section 3 describes the main use cases of utilizing Ontologies in Software-Engineering-Meta-Research; Section 4 is about ontology-based systems in scientific search. Finally, in section 5 are made the conclusions regarding to the whole seminar work.               