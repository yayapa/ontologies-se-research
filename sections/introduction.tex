%% LaTeX2e class for seminar theses
%% sections/content.tex
%% 
%% Karlsruhe Institute of Technology
%% Institute for Program Structures and Data Organization
%% Chair for Software Design and Quality (SDQ)
%%
%% Dr.-Ing. Erik Burger
%% burger@kit.edu
%%
%% Version 1.0.2, 2020-05-07

\section{Introduction}
\label{ch:Introduction}
The term ontology origins from philosophy. Its etymology gives us the direct definition i.e. a science about being. \flqq{ More broadly, it studies concepts that directly relate to being, in particular becoming, existence, reality, as well as the basic categories of being and their relations.}\frqq[cite wiki]. But how foes it relate to Computer Science? As we know, Information and Computer science are very precious and cannot admit a vague descriptions, as it is done in philosophy. Thus, ontology in Computer science is a science about\flqq{representation, formal naming and definition of the categories, properties and relations between the concepts, data and entities that substantiate one, many or all domains of discourse.}\frqq[cite wiki]. This means simply the supplement of a common language, in order to facilitate general understanding of knowledge in machine way.    
   
\begin{itemize}
	\item The definition of Ontology in Philosophy and Computer Science
	\item The definition of Meta Research
	\item Above-named terms Ontology and Meta-research in Software Engineering(SE)
	\item Help of ontology-based systems in scientific search
\end{itemize}

Generally, the content (what it will be about) of the following proseminar work is summed up.