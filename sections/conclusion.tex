%% LaTeX2e class for seminar theses
%% sections/conclusion.tex
%% 
%% Karlsruhe Institute of Technology
%% Institute for Program Structures and Data Organization
%% Chair for Software Design and Quality (SDQ)
%%
%% Dr.-Ing. Erik Burger
%% burger@kit.edu
%%
%% Version 1.0.2, 2020-05-07

\section{Conclusion}
\label{ch:Conclusion}

In this seminar work, based on extensive references that were obtained by using both reference-based and data-based search, the knowledge about different concepts have been received, summed up, presented and analyzed. Among them were ontology in computer science, different ontology languages, their expediency and limitations. Then, the term Meta-research and the corresponding categorization, that is used during the whole work, have been presented. Combining these two terms in Software Engineering, namely the using and application of ontologies in Software-Engineering-Meta-Research have been investigated and a part of the current state-of-the-art in this field has been reported. Besides, the presented ontologies have been compared and analyzed by extracting their advantages and disadvantages. Moreover, during the search processes the affinity of the above-named terms with Semantic Web and Semantic search engine were found out and, thus, it was decided to investigate the ontology-based systems in scientific search. For this purpose the necessary foundations in this field have been briefly set up and, then, the diversity of existing scientific ontologies have been investigated, presented and tried to classify. To the very end, the requirements for the present scientific ontologies integrated in the Open Research Knowledge Graph (ORKG) have been discussed and also, other approaches that ORKG have been introduced.

Furthermore, the aim of this seminar work was not only to find necessary papers and references, summarize and report about them, but also to categorize, compare, and analyze them and, finally, to point out to possible perspectives and future works. 

It has been shown, how using ontologies has changed many scientific fields such as Mathematics, Medicine and, especially Computer Science. In my pinion, the using of  ontologies in Software-Engineering-Meta-Research has already become very important by playing a major role in the research field. Moreover, I suppose it will become more important and manage to change Software Engineering for the better.       